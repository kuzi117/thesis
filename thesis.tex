% University of Alberta Example Thesis
% By the Rogue's Gallery, Department of Computing Science
% Last updated September 15, 2017

% Note: You will probably have to edit the uathesis.sty file
%       to comment or uncomment bits and pieces
%       (e.g. co-supervisor, externals, etc)

% A workaround to allow relative paths in included subfiles
% that are to be compiled separately
% See https://tex.stackexchange.com/questions/153312/subfiles-inside-a-subfile-using-relative-paths
\providecommand{\main}{.}
\documentclass[12pt]{report}          % for default format

%%%%%%%%%%%%%%%%%%%%%%%%%
% Package dependencies  %
%%%%%%%%%%%%%%%%%%%%%%%%%
\usepackage[utf8]{inputenc}
\usepackage{subfiles}
\usepackage[titletoc]{appendix}
\usepackage{amssymb,amsmath}
\usepackage{url}
% For cross-references between files
% https://tex.stackexchange.com/questions/77774/undefined-control-sequence-when-cross-referencing-with-xr-hyper
\usepackage{nameref,zref-xr}
\zxrsetup{toltxlabel}
\usepackage{hyperref} % See https://en.wikibooks.org/wiki/LaTeX/Hyperlinks#Customization
\hypersetup{
  colorlinks = true, %Colours links instead of ugly boxes
  urlcolor = {blue!80!black}, %Colour for external hyperlinks
  linkcolor = {green!35!black}, %Colour of internal links
  citecolor = {blue!50!black} %Colour of citations
}
\usepackage{tabulary} % Better text wrapping in tables. See https://en.wikibooks.org/wiki/LaTeX/Tables
\usepackage{rotating} % For the 'sidewaystable' environment. See https://en.wikibooks.org/wiki/LaTeX/Rotations
\usepackage{multirow} % For multirow/multicolumn cells in a table. See https://en.wikibooks.org/wiki/LaTeX/Tables#Columns_spanning_multiple_rows
\usepackage{graphicx}
%\usepackage{subcaption} % A package that can be used to create sub-figures
\usepackage[inline]{enumitem} % for more control over enumerate
% \usepackage{mdwlist}	% the 'note' environment; disabled because it conflicts with enumerate*. If we need note, we'll need to add something else.
\usepackage{xspace}
\usepackage{xcolor}
\usepackage{csquotes} % Also used with biblatex
\usepackage{xmpincl} % Seems to be needed when converting to PDF/A,
                     % particularly on Macs

% (Added by Braedy)
% \usepackage{layouts}
\usepackage{algorithm}
\usepackage[noend]{algpseudocode}
\usepackage{listings} % code listings
\lstset{
  aboveskip=\bigskipamount,
  % belowskip=\medskipamount,
  abovecaptionskip=\bigskipamount,
  language=[11]C++,
  numbers=left,
  xleftmargin=1.5em,
  tabsize=2,
  numberbychapter=true,
  captionpos=b
}
\renewcommand\lstlistlistingname{List of Listings}
\usepackage{soul} % strikethrough
\usepackage{tikz}
\usetikzlibrary{decorations.pathreplacing}
\usepackage[all]{hypcap} % Fixes hyperref linking to the bottom of a figure.

% Change Statex to take optional extra indentation parameter.
\makeatletter
\let\OldStatex\Statex
\renewcommand{\Statex}[1][3]{%
  \setlength\@tempdima{\algorithmicindent}%
  \OldStatex\hskip\dimexpr#1\@tempdima\relax}
\makeatother

%%%%%%%%%%%%%%%
% biblatex    %
%%%%%%%%%%%%%%%
% (Added by Bernard Llanos)
% biblatex is intended to be the successor to BibTeX
% (https://en.wikibooks.org/wiki/LaTeX/Bibliography_Management#biblatex)
\usepackage[%
backend=bibtex,%
style=ieee,%
backref=false,%
sortcites=true,sorting=nyt,%
mincitenames=1,maxcitenames=2%
]{biblatex}
% `backref=true` adds back references - Links to the in-text citations from
% the corresponding bibliography entries. Back references are not mentioned
% in thesis guidelines, but are, in my opinion, helpful for reading and editing.

\usepackage[american]{babel}

% The following macro will put back references on the right edge of the page
% (https://tex.stackexchange.com/questions/149009/biblatex-pagebackref-reference-in-the-flush-right-margin)
\renewbibmacro*{pageref}{%
   \iflistundef{pageref}
     {\renewcommand{\finentrypunct}{\addperiod}}
     {\renewcommand{\finentrypunct}{\addspace}%
      \printtext{\addperiod\hfill\rlap{\hskip15pt\colorbox{blue!5}{\scriptsize\printlist[pageref][-\value{listtotal}]{pageref}}}}}}

\addbibresource{\main/refs.bib}

%%%%%%%%%%%%%%%%%%%%%%%%%%%%%%%%%%%%
% Glossary and acronyms (optional) %
%%%%%%%%%%%%%%%%%%%%%%%%%%%%%%%%%%%%
\usepackage[nogroupskip,nonumberlist,nopostdot,acronym,toc]{glossaries}

\setacronymstyle{long-short-desc}
\setglossarystyle{altlistgroup}

% A simple, but limited way to produce a sorted glossary
% Other options are described in the 'glossaries' Beginner's Guide (https://ctan.org/pkg/glossaries)
\makenoidxglossaries % use TeX to sort

% Glossary entries will be loaded from a separate file
\loadglsentries{\main/tex/glossary}

%%%%%%%%%%%%%%%%%%%%%%%%
% Thesis style package %
%%%%%%%%%%%%%%%%%%%%%%%%
\usepackage{\main/uathesis}  % Earlier version says this should be last package
                     % imported. Never checked if this is still true.
                     % Having this second last before the next one seems fine.

%%%%%%%%%%%%%%%%%%%%%%%%%%%%
% Shorthands               %
%%%%%%%%%%%%%%%%%%%%%%%%%%%%

% From the CVPR paper template (http://cvpr2017.thecvf.com/submission/main_conference/author_guidelines)
% Add a period to the end of an abbreviation unless there's one
% already, then \xspace.
\makeatletter
\DeclareRobustCommand\onedot{\futurelet\@let@token\@onedot}
\def\@onedot{\ifx\@let@token.\else.\null\fi\xspace}
\makeatother

\def\eg{\emph{e.g}\onedot} \def\Eg{\emph{E.g}\onedot}
\def\ie{\emph{i.e}\onedot} \def\Ie{\emph{I.e}\onedot}
\def\cf{\emph{c.f}\onedot} \def\Cf{\emph{C.f}\onedot}
\def\etc{\emph{etc}\onedot} \def\vs{vs\onedot}
\def\wrt{w.r.t\onedot} \def\dof{d.o.f\onedot}
\def\etal{\emph{et al}\onedot}

% Set the automatic text for autoref.
\def\chapterautorefname{Chapter}
\def\sectionautorefname{Section}
\def\subsectionautorefname{Section}
\def\tableautorefname{Table}
\def\algorithmautorefname{Algorithm}

% Set up autoref convenience commands.
\newcommand{\rcha}[1]{\autoref{cha:#1}}
\newcommand{\rsec}[1]{\autoref{sec:#1}}
\newcommand{\rtab}[1]{\autoref{tab:#1}}
\newcommand{\rfig}[1]{\autoref{fig:#1}}
\newcommand{\rlst}[1]{\autoref{lst:#1}}
\newcommand{\ralg}[1]{\autoref{alg:#1}}
\newcommand{\rdef}[1]{\autoref{def:#1}}

% Other convenience commands.
\newcommand{\code}[1]{\lstinline[basicstyle=\ttfamily,breaklines=true]|#1|}
\newcommand{\mat}[3]{$\underset{(#2 \times #3)}{#1}$}
\newcommand{\matmul}[3]{$\underset{(#1 \times #2)}{A} \times \underset{(#2 \times #3)}{B} = \underset{(#1 \times #3)}{C}$}

% Reviewer nnotation commands.
\newif\ifcomments
\commentstrue % Comment out or set \commentsfalse to disable comments.
\newcommand{\bk}[1]{\ifcomments\textcolor{blue}{([Braedy]: #1)}\fi}
\newcommand{\jc}[1]{\ifcomments\textcolor{orange}{([João]: #1)}\fi}
\newcommand{\nelson}[1]{\ifcomments\textcolor{red}{([Nelson]: #1)}\fi}
\newcommand{\guido}[1]{\ifcomments\textcolor{cyan}{([Guido]: #1)}\fi}
\newcommand{\kit}[1]{\ifcomments\textcolor{purple}{([Kit]: #1)}\fi}
\newcommand{\jm}[1]{\ifcomments\textcolor{olive}{([JM]: #1)}\fi}
\newcommand{\del}[1]{\ifcomments\textcolor{gray}{\st{#1}}\fi}
\newcommand{\new}[1]{\textcolor{green}{#1}}
\newcommand{\ed}[1]{\textcolor{violet}{#1}}
\newcommand{\todo}[1]{\textcolor{magenta}{(#1)}}

%%%%%%%%%%%%%%%%%%%%%%%%%%%%%%%%%%%%%%%%%%%
% Title page and Table of Contents Tweaks %
%%%%%%%%%%%%%%%%%%%%%%%%%%%%%%%%%%%%%%%%%%%

%% Correct title for TOC
\renewcommand{\contentsname}{Table of Contents}

% Fill in the following
\title{Thesis Title} % Title can't use formulae, symbols, superscripts, subscripts, greek letters, etc. all of which should be replaced with word substitutes
\author{Braedy Kuzma}

\degree{\MSc}
%\degree{\PhD} % uncomment respective degree

\dept{Computing Science}  % Write Computing Science or Civil Engineering.

% If you have a specialization, uncomment the following line and enter it below.
% It must correspond with what it says on Bear Tracks
% (Academics>My Academics>Graduation). If, like most people, you don't have one,
% just leave it commented.
%\field{Specialization Field}

% Put the year that you submitted your thesis below
\submissionyear{\number2020}

%%%%%%%%%%%%%%%%%%%%%
% Document Content  %
%%%%%%%%%%%%%%%%%%%%%

% This is a modular document.
% The 'subfiles' package allows you to typeset the included
% documents separately from the main document, so that you
% can view only pieces of the thesis at a time.
% See https://en.wikibooks.org/wiki/LaTeX/Modular_Documents
%
% Subfiles that contain references: You can just run
% `biber subfilename` on them when compiling them individually.
% There is no need to make them reference the bibliography database
% 'refs.bib', as they inherit the reference from this file.

\newcommand{\onlyinsubfile}[1]{#1}
\newcommand{\notinsubfile}[1]{}

\begin{document}

\renewcommand{\onlyinsubfile}[1]{}
\renewcommand{\notinsubfile}[1]{#1}

\preamblepagenumbering % lower case roman numerals for early pages
\titlepage % adds title page. Can be commented out before submission if convenient

\subfile{\main/tex/abstract.tex}

% Spacing for the rest of the douc
% \singlespacing
% \onehalfspacing
\doublespacing
% \truedoublespacing

% A preface is not always required
\subfile{\main/tex/preface.tex}

% Below are the dedication page and the quote page. FGSR requirements are not
% clear on if you can have one of each or just one or the other. They do say to
% ask your supervisor if you should have them at all.
%
% The CS Department links to a comparison of pre- and post-Spring 2014 thesis
% guidelines (https://www.ualberta.ca/computing-science/graduate-studies/current-students/dissertation-guidelines)
% The comparison document lists an optional dedication page, but no quote page.

\subfile{\main/tex/dedication.tex}
\subfile{\main/tex/quote.tex}

\subfile{\main/tex/acknowledgements.tex}

\singlespacing % Flip to single spacing for table of contents settings
               % This has been accepted in the past and shouldn't be a problem
               % Now the table of contents etc.

\tableofcontents\newpage
\addcontentsline{toc}{chapter}{List of Tables}
\listoftables\newpage
\addcontentsline{toc}{chapter}{List of Figures}
\listoffigures\newpage
\addcontentsline{toc}{chapter}{List of Algorithms}
\listofalgorithms\newpage
\addcontentsline{toc}{chapter}{List of Listings}
\lstlistoflistings

% minimal support for list of plates and symbols (Optional)
%\begin{listofplates}
%...            % you are responsible for formatting this page.
%\end{listofplates}
%\begin{listofsymbols}
%...            % You are responsible for formatting this page
%\end{listofsymbols}

% A glossary of terms is also optional
\glsunsetall % Stop "first time" expansion in the glossary.
\printnoidxglossaries
\glsresetall % All references to glossary items are now expanded like it's the first time.

% The rest of the document has to be at least one-half-spaced.
% Double-spacing is most common, but uncomment whichever you want, or
% single-spacing if you just want to do that for your personal purposes.
% Long-quoted passages and footnotes can be in single spacing
\doublespacing
% \truedoublespacing
% \singlespacing
% \onehalfspacing

\setforbodyoftext % settings for the body including roman numeral numbering starting at 1

%  ... The bulk of your magnificient thesis  goes here ...
%  hopefully more than two lines! Use standard Latex sectioning commands
%  like \chapter etc. End with the bibliography
% See FGSR requirements for any additional requirements on the body

\subfile{\main/tex/introduction.tex}
\subfile{\main/tex/background.tex}
\subfile{\main/tex/related.tex}
\subfile{\main/tex/matmul.tex}
\subfile{\main/tex/mma.tex}
\subfile{\main/tex/method.tex}
\subfile{\main/tex/evaluation.tex}
% \subfile{\main/tex/related.tex} % Possibility.
\subfile{\main/tex/conclusion.tex}

% Renaming the bibliography: See http://tex.stackexchange.com/questions/12597/renaming-the-bibliography-page-using-bibtex
\renewcommand\bibname{References}
\clearpage\addcontentsline{toc}{chapter}{\bibname}
     %add the above line to get "References" in the table of contents.
%
\singlespacing % optional;  Bibliography is better in single spacing
               %            but you may choose different
               %            Don't use \singlespacing if your thesis
               %            is already in single spacing
%
\printbibliography

\doublespacing

% Appendices are optional.
%
% The standard `\appendix` macro will not put the word "appendix" before
% table of contents entries.
% See https://tex.stackexchange.com/questions/44858/adding-the-word-appendix-to-table-of-contents-in-latex
% \begin{appendices}
% \subfile{\main/tex/background.tex}
% \end{appendices}

\end{document}
