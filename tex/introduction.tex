% A workaround to allow relative paths in included subfiles
% that are to be compiled separately
% See https://tex.stackexchange.com/questions/153312/subfiles-inside-a-subfile-using-relative-paths
\providecommand{\main}{..}
\documentclass[\main/thesis.tex]{subfiles}

\begin{document}

\chapter{Introduction}
\label{cha:intro}

Matrix multiplication is a critical basic operation in many \gls{hpc} and AI workloads.
Given the surge of popularity in AI and the growing scale of \gls{hpc} tasks and simulations, the optimisation of matrix multiplication can mean an essential reduction in computing time.
Waugh and McIntosh-Smith demonstrate that for small thread counts, \gls{gemm} operations often dominate execution time in a set of benchmarks found to be representative of current workloads~\autocite{waugh2020use}.
While multithreading reduces these functions' contribution to overall runtime significantly, they conclude that, in a soon-to-be-exascale future, applications will adapt to use more \gls{gemm} operations, expanding this portion considerably.

At the center of our capability to handle these expanding workloads is the \gls{matrix engine}.
\Glspl{matrix engine} are a new accelerator facility, found in the most recent generation of \glspl{cpu}, that focuses on accelerating matrix multiplication.
Currently, acceleration is done via external accelerators or via vector extensions to an \gls{isa}.
An external accelerator may be an \gls{asic} specialised to the task (\eg Google\texttrademark{}'s \gls{tpu}\texttrademark{}~\autocite{abadi2016tensorflow}), a \gls{gpu}, or \ifglsused{fpga}{an}{a} \gls{fpga}.
Vector extensions are part of the movement towards the \gls{simd} computing paradigm where multiple pieces of data are used and produced by a single instruction (see \rsec{simd}).
An advantage of vector extensions in relation to external accelerators is that a vector extension can access data through the same memory hierarchy used by the \gls{cpu} while most external accelerators require data transfer through an interconnect with slower transfer speeds.

The current state of the art when working with matrix operations is to choose one of several libraries that implement the \gls{blas} interface (\eg OpenBLAS~\autocite{xianyi2012model}, IBM's ESSL~\autocite{ibm2021engineering}, Intel\textsuperscript{\textregistered{}}'s MKL~\autocite{wang2014intel,intel2021accelerate}, Nvidia\textsuperscript{\textregistered{}}'s cuBLAS~\autocite{nvidia2021cublas}).
These libraries can provide incredible speedups and, in parallel architectures, automatic parallelisation: a very attractive feature for large workloads.
It has been a natural extension of these libraries to include the usage of matrix engines when targeting \glspl{cpu}.
However, the development of most of these libraries rely on difficult-to-maintain assembly programs and their usage imposes extra requirements on systems.

All high-performance implementations of \gls{blas}-like libraries have handwritten assembly kernels at their core, though the extent of the kernel varies~\autocite{zee2016blis}.
Nevertheless, each of these kernels must be produced and hand tuned for each new \gls{cpu} that the library needs to support.
Creating these kernels requires engineers who are extremely knowledgeable about the \gls{isa} and the architectural details of a target \gls{cpu}.
Thus, maintaining the code surrounding such a kernel as well as the kernel itself requires significant ongoing effort as developers try to obtain greater performance from the implementation.

Additionally, using a library means that user code now has an external dependency.
In applications where performance and correctness are critical and must be tightly controlled by the developer, using libraries may be an impossibility.
Similarly, while finding a \gls{blas} implementation on major platforms is often quite easy, some target platforms may not have an available implementation with which this dependency can be fulfilled~\autocite{zee2016blis}.
This is a direct product of the manual kernel porting difficulties described above.

Therefore, as a counterpoint, a method of transparently accelerating matrix multiplication via a compiler-only path is important for portability and maintainability.
When a matrix-multiplication kernel is created as part of a compiler, it no longer needs to be written in assembly.
Relaxing this requirement means that kernels become \emph{\gls{cpu} agnostic} rather than \emph{\gls{cpu} dependent}.
It will not be \emph{\gls{isa} agnostic} because a kernel, to obtain best performance, will still need to be written in terms of an \gls{isa}'s vector extension, tying it to the architecture, but it will not be tied to a specific version of the \gls{isa}.
The change to \gls{cpu} agnostic does, however, mean that optimisations that are based on architectural details such as register count or available functional units, which would typically be done manually in a handwritten kernel, can now be automatically performed and tuned to the \gls{cpu} by the compiler.
This automation means that an appropriately written and parameterised kernel will also still be optimal in future hardware iterations.

Kernels written as part of a compiler are also implicitly subject to all of the optimisations available in the compiler, now and in the future.
Currently, these include optimisations such as loop interchange, blocking, and unrolling which are known to have significant effects on matrix multiplication speed~\autocite{velkoski2014performance,allen1984automatic,fabeiro2016writing,schreiber1990automatic}.
For example, Velkoski \etal demonstrate a $\sim 2.5 \times$ speedup using unrolling alone for certain parameters; interchange and blocking have a much greater effect on memory-heirarchy usage, providing greater benefits as the operation size grows.
In the backend, improvements to processes such as register allocation and instruction scheduling (reordering/pipelining) will be retroactively available to these kernels as well, completely transparently to the user.

Furthermore, a crucial optimisation, loop fusion, is impossible with the current paradigm of library function calls.
Mathematically, consecutive matrix operations (\eg $D = ABC$) can be fused so as to perform both computations at once instead of producing a temporary result (\ie $\text{T}=AB;D=TC$).
Fusing operations in this manner significantly reduces the memory movements required by removing the need to store and reload the whole-matrix temporary value.
Moreover, given sufficiently large matrices, it is likely that portions of the matrix have already been forced out of the cache.
A compiler has the foresight and tools to fuse these operations and then further optimise the resulting code.

A compiler-only solution for matrix multiplication, while benefitting from preexisting features of a compiler, nevertheless requires
\begin{enumerate*}[itemjoin={{; }}, itemjoin*={{; and }}, label={}, afterlabel={}, after={.}]
  \item knowledge of the compiler framework, target operation, and the target architecture
  \item forethought based on the constraints derived from this knowledge
  \item an actionable design built from this forethought
\end{enumerate*}
As in a handwritten solution, required knowledge includes an understanding of the matrix engine and vector extensions of the target architecture as well as how they interact with each other and with the memory hierarchy.
However, because the kernel is written at a higher level within the compiler, it counters many of the issues associated with the handwritten assembly version.
Authors are now more productive by only having to focus on writing the kernel: registers and the data in them are now automatically managed, the instruction schedule can be optimised by the compiler to hide latencies, loops are unrolled according to architecture capacity, and more.
The produced kernels are more portable, functioning on \glspl{cpu} with the same \gls{isa} (in some cases cross-\gls{isa}) with much of the infrastructure being portable between architectures.
Finally, maintenance is also easier: higher level languages are more readable with less expertise required to interact with them.

The method of delivering the kernel must also be considered.
Libraries are the currently preferred method of acceleration and therefore any replacement method should aim to require less effort if it hopes to be adopted.
Library download and installation is oftentimes a simple barrier to overcome for the average user whereas the majority of users are loath to find and benchmark combinations of compiler flags in an effort to optimise their program.
They would prefer instead to assume that the preset, curated optimisation levels (\ie \code{-O1}, \code{-O2}, \code{-O3}) are sufficient.
Therefore, a compiler-only solution should strive to be as transparent as possible when it comes to enabling easy adoption.

\end{document}
