% A workaround to allow relative paths in included subfiles
% that are to be compiled separately
% See https://tex.stackexchange.com/questions/153312/subfiles-inside-a-subfile-using-relative-paths
\providecommand{\main}{..}
\documentclass[\main/thesis.tex]{subfiles}

\begin{document}

\chapter{Introduction}
\label{cha:intro}

\nelson{Everything that you have in this chapter should move to the background chapter. The introduction should explain the problem that you are solving. Re-read the introductions to the KernelFarer paper and to the PACT paper to get your mind into what an introduction should say. You can either talk about the same ideas and white your own, or you can reuse some of that text if you have a paragraph saying that the text appears in those multi-author manuscripts.}

\nelson{The introduction should contain a clear problem statement or thesis statement. Someone reading only the introduction should be able to clearly state the contribution of a thesis.}


\section{Contributions}
This work focuses on how \gls{mma} can improve matrix multiplication on PowerPC
This work addresses multiple questions of interest when considering MMA.
\begin{enumerate}
  \item
    Does an \gls{mma} kernel perform better than the vectorised equivalent?
  \item
    Does \gls{mma} offer greater benefits when using different datatypes?
\end{enumerate}

\end{document}
