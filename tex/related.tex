% A workaround to allow relative paths in included subfiles
% that are to be compiled separately
% See https://tex.stackexchange.com/questions/153312/subfiles-inside-a-subfile-using-relative-paths
\providecommand{\main}{..}
\documentclass[\main/thesis.tex]{subfiles}

\begin{document}

\chapter{Related Work}
\label{cha:related}
The sections below present work related to separate facets of this thesis.
Each of the sections contains text derived from work currently under revision~\autocite{kuzma2021fast}.

\section{Matrix Multiplication}
In a seminal work, Goto and van de Geijn detail a layered approach to improve cache and vector-register utilization on \glspl{cpu}~\autocite{goto2008anatomy}.
Using this approach, modern linear-algebra libraries, such as Eigen and OpenBLAS, achieve high performance on \gls{hpc} workloads.
Goto and van de Geijn show that modelling both L2 cache and \gls{tlb} --- and not only L1 cache as considered earlier --- is crucial for cache performance.
Their work is seminal because it publicly explained practical strategies for optimal cache and vector register utilization on \glspl{cpu}; these strategies were previously only available in proprietary libraries.
The layered strategy features two stages:
\begin{enumerate*}[itemjoin={{; }}, itemjoin*={{; and }}, label=\textbf{(\arabic*)}, after={.}]
  \item blocking input matrices and packing tiles of these blocks in such a way that tiles lay in main memory in the order that they will be accessed
  \item computing a small \gls{gemm} at the register level
\end{enumerate*}
Kuzma \etal~\autocite{kuzma2021fast} is the first work to create a compiler-only code generation for the layered approach and adapts blocking, tiling, and packing to create a data layout that is suitable for computing with \gls{mma} and to also improve utilization of the L3 cache.
This thesis is an in-depth presentation of the method behind \textbf{(2)}.

Gareev \etal~\autocite{gareev2018high} implement tiling and packing within Polly~\autocite{grosser2011polly,grosser2012polly} without the need for an external library or automatic tuning.
Their approach with a hand-optimized x86 SSE kernel reports performance on par with BLIS on Intel Sandy Bridge.
When not relying on an assembly kernel, their pass uses the default \gls{llvm} \glslink{vectorisation}{vectoriser} that delivers only a small speedup over na\"ive code.
The work in this thesis presents a compiler-only kernel that can replace Gareev \etal's SSE kernel or the default \gls{llvm} \glslink{vectorisation}{vectoriser} on \gls{power10}.
Furthermore, this work is more modular and can be reused by new passes or other code generation paths as drop-in inner-kernel.

Uday Bondhugula presents an implementation of the \gls{blas} strategy within the emerging MLIR framework~\autocite{bondhugula2020high}.
He demonstrates that blocking, packing, register tiling, and unroll/jam yields code that achieves 34\% of OpenBLAS' performance on Intel's Coffee Lake~\autocite{bondhugula2020high}.
Bondhugula also implemented a custom \gls{vectorisation} pass to replace the default \gls{llvm} \glslink{vectorisation}{vectoriser} thus reaching 91\% of the performance of OpenBLAS.
This work also shows the weakness of the default \gls{llvm} \glslink{vectorisation}{vectoriser}.
As well, because \gls{llvm} \gls{ir} is a dialect within MLIR, using \gls{llvm} \gls{ir} \glspl{intrinsic} is inherently an accessible method of implementing an inner kernel in MLIR.

Carvalho \etal introduce KernelFaRer, a robust pattern recognition system that can identify matrix-multiplication patterns in the \gls{llvm} \gls{ir} level and can replace the matrix multiplication with library calls~\autocite{carvalho2021kernelfarer}.
While this approach can lead to speedups on the order of 1000s in comparison with non-optimized code, it has the drawback of requiring the use of libraries into a computer system that may not have them installed.
Moreover, their experimental results indicate that, for smaller matrices, the overhead of invoking functions in the libraries leads to performance degradations.
The solution in this thesis is orthogonal to KernelFaRer.
The pattern recognition in KernelFaRer can identify \gls{gemm} kernels at the intermediate-level representation and then replace the inner-most kernel with the solution presented here.

\section{Code Generation for New Hardware}
When presenting the ILLIAC IV, one of the first SIMD machines, Barnes \etal advocated that data parallelism would be crucial for progress~\autocite{barnes1968illiac}, citing matrix operations as a critical target~\autocite{kuck1968illiac}.
Nearly 50 years later, Barnes' direction culminated in the inclusion of vector extensions in all mainstream \glspl{cpu} such as \gls{ibm}'s \gls{vsx}~\autocite{PowerISA}, Intel's AVX-512~\autocite{IntelISA}, and ARM's NEON/\gls{sve}~\autocite{ArmISA}.
Although fast \gls{vectorisation} is powerful, matrix-multiplication performance could be improved further with specialized hardware units.
This possibility is now realized with the introduction of what Domke \etal have dubbed ``matrix engines''~\autocite{domke2021matrix}, now available in \gls{ibm}'s \gls{mma}~\autocite{PowerISA}, Intel's \gls{amx}~\autocite{IntelISA}, and ARM's NEON/\gls{sve}~\autocite{ArmISA}.
This thesis, in the same vein as Barnes \etal, focuses on bringing high performance to new hardware facilities.

\section{Other Matrix Engines}
The advent of the ``general purpose'' \glspl{gpu} quickly saw study and performance analysis of matrix computations~\autocite{larsen2001fast,fatahalian2004understanding}.
This evolved into implementations of matrix multiplications on \glspl{gpu}: manually~\autocite{li2011strassens}, through libraries like \gls{blas}~\autocite{nath2011accelerating}, and through frameworks such as DistME~\autocite{han2019distme}.
Matrix multiplication is also central to the design of hardware for tensor-operation acceleration such as Google's Tensor Processing Unit~\autocite{jouppi2017datacenter}, Nvidia's Tensor Core~\autocite{markidis2018nvidia}, and Huawei's Cube Unit~\autocite{liao2019davinci}.
Performance evaluations of \gls{gemm} in tensor hardware are difficult to find because the studies of these devices focus on the benchmarking of various flavours of neural network~\autocite{jouppi2017datacenter,wang2019benchmarking}.
Matrix-multiplication acceleration in standalone accelerators is not the focus of this thesis.

It is difficult to find works explicitly discussing the outer product because kernels are often described in terms of Goto and van de Geijn's work.
However, a work by Naohito Nakasato~\autocite{nakasato2011fast} and another by Wu Jing and Joseph Jaja~\autocite{wu2016achieving} both explicitly mention the use of the outer product to compute matrix multiplication.
Both of these works use a \gls{gpu} for their implementation while this work focuses on the emerging field of on-chip matrix engines.

Yu \etal implement a rank-one update algorithm on the \gls{gpu}, but their research focus is improving the performance of bilinear pooling~\autocite{yu2020toward}.
Their work aims to replace another method of singular value decomposition via an iterative method requiring many matrix-matrix multiplications by using the outer product.
This work presented in this thesis aims to improve matrix-matrix multiplication on the \gls{cpu} using rank-$r$ updates.

Pal \etal introduce an outer product accelerator designed for sparse-matrix multiplication~\autocite{pal2018outerspace}.
Their design, called OuterSPACE, is an external accelerator focused on alleviating the memory issues associated with sparse matrix-matrix multiplication.
The algorithm presented in this thesis uses \gls{mma} as an accelerator for dense matrix-matrix multiplication.
Despite the focus of this thesis on matrix multiplication, \gls{mma} is not at all limited to dense operations.
Work by Gu \etal develops a memory-efficient algorithm for sparse matrix-matrix multiplication using the outer product that may prove to be an excellent match with \gls{mma}~\autocite{gu2020bandwidth}.

\section{Performance Evaluation}
Robust performance benchmarking is critical for the evaluation of vector extensions.
While there is extensive performance evaluation of matrix multiplication on vector extensions for Intel architectures~\autocite{hassan20161performance,hemeida2020optimizing,alappat2020understanding}, to the best of the author's knowledge, similar studies do not exist for the PowerPC or ARM platforms.
Moreover, the introduction of matrix engines is recent in all platforms and therefore only simulated or theorized performance estimates exist for \gls{amx}, \gls{sve}, or \gls{mma}~\autocite{poenaru2020evaluating,domke2021matrix}.
Therefore, this work is among the first to present performance evaluation of a matrix engine on actual hardware.

\end{document}
