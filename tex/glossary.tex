% A
\longnewglossaryentry{altivec}{name={AltiVec},text={AltiVec}}{
  A \gls{simd} instruction set designed for use with single-precision floating point and integer values.
}
\newacronym[
  description={
    An \gls{integrated circuit} built for a specific use instead of general purpose use.
  }
]{amx}{AMX}{Advanced Matrix Extension}
\newacronym[
  description={
    The matrix engine extension in x86.
  }
]{asic}{ASIC}{Application-Specific Integrated Circuit}

% B
\longnewglossaryentry{basic block}{name={Basic Block}, text={basic block}}{
  A straight-line code sequence where only the final statement is some sort of control flow and only the first statement may receive control flow.
  Alternatively, an atomic block of statements where if the first statement is executed, all statements in the block must be executed.
  Basic blocks form the nodes in a control flow graph.
}
\newacronym[
  description={
    Originally a set of highly-optimised, low-level routines used as building blocks in many larger linear algebra operations~\autocite{lawson1979basic}.
    It has since developed into a standard interface for higher (matrix-matrix) and lower level (scalar/vector, vector/vector) linear algebra operations with many open source and proprietary implemenations.
  }
]{blas}{BLAS}{Basic Linear Algebra Subprograms}
\longnewglossaryentry{builtin}{name={Builtin}, text={builtin}}{
  A function callable from a higher-level language (\eg C/C++) which encapsulates a well-known functionality with well-defined semantics.
  Such a function may be implemented in header files, as language-specific keywords, or conditionally available functions.
}
\longnewglossaryentry{broadcast}{name={Broadcast}, text={broadcast}}{
  The act of duplicating a scalar value to all lanes in a vector register.
}

% C
\newacronym[
  description={
    The main component of a computer which interacts with and controls all other subcomponents according to the instructions contained in a program.
  }
]{cpu}{CPU}{Central Processing Unit}
\longnewglossaryentry{clang}{name={Clang}, text={Clang}}{
  The C/C++ frontend available as part of the \gls{llvm} project.
}
\newacronym[
  description={
    An optimisation which searches for identical expressions and potentially replaces them with a single variable~\autocite{cocke1970global,rosen1988global}.
  }
]{cse}{CSE}{Common Subexpression Elimination}
\longnewglossaryentry{compilation unit}{name={Compilation Unit}, text={compilation unit}}{
  See \gls{translation unit}.
}
\newacronym[
  description={
    The number of CPU clock cycles required to complete an instruction.
    Often used as a performance metric where it refers to the ratio of total instructions executed to total clock cycles.
  }
]{cpi}{CPI}{Cycles Per Instruction}

% F
\newacronym[
  description={
    An \gls{integrated circuit} which is configurable post-production via reconfigurable interconnects between logic blocks which perform operations.
  }
]{fpga}{FPGA}{Field-Programmable Gate Array}
\newacronym[
  description={
    An instruction which performs the operation $w = x \times y + z$ often used to perform accumulation operations as in $w \mathrel{+}= x \times y$.
    \gls{simd} equivalents exist, performing the same operation per element of the operand vectors.
  }
]{fma}{FMA}{Fused Multiply-Add}

% G
\newacronym[
  description={
    A more general form of matrix multiplication with a scaling factor for accumulation and multiplication, \ie $C' = \alpha AB + \beta C$ where $C'$ is the new value of $C$ after accumulation.
  }
]{gemm}{GEMM}{General Matrix-Matrix Multiplication}
\newacronym[
  description={
    Originally designed to accelerate image creation for display devices, it was soon repurposed into a general purpose, high performance compute device.
  }
]{gpu}{GPU}{Graphics Processing Unit}

% H
\newacronym[
  description={
    The area of computing concerned with performing large scale computations in the most efficient way possible through the minimisation of required time and resources.
  }
]{hpc}{HPC}{High Performance Computing}

% I
\newacronym[
  description={
    A multinational technology company with a long history in the computing domain.
  }
]{ibm}{IBM}{International Business Machines}
\newacronym[
  description={
    A language-agnostic representation of a program encoding all of the semantics of the original program and potentially annotated with debugging and optimisation information.
  }
]{ir}{IR}{Intermediate Representation}
\longnewglossaryentry{integrated circuit}{name={Integrated Circuit},text={integrated circuit}}{
  A set of electronic circuits built into a semiconductor, typically silicon.
  These circuits are set in place and cannot be changed.
}
\longnewglossaryentry{intrinsic}{name={Intrinsic},text={intrinsic}}{
  A generic function within an \gls{ir} which encapsulates a well-known functionality with well-defined semantics.
  The function will eventually be \glslink{lowering}{lowered} to either a call to an existing function or inline code.
  Some compilers (\eg Microsoft Visual C++) may use this term to refer to a \gls{builtin}.
}
\newacronym[
  description={
    The abstract model of a computer, including registers, supported datatypes, instructions, \etc.
  }
]{isa}{ISA}{Instruction Set Architecture}

% L
\longnewglossaryentry{lane}{name={Lane},text={lane}}{
  A way to refer to an element of a vector register or operation derived from the way circuit pathways leave and enter vector functional units in groups corresponding to an element, like the lanes of a roadway.
  The number of lanes in a vector register, which has constant width, changes based on element width, \ie a 128 bit register with 32 bit elements has four lanes, with 64 bit elements it has two lanes.
}
\longnewglossaryentry{linking}{name={Linking},text={linking}}{
Achieved through a program called a ``Linker'', this is typically the final step in the compilation process.
The procedure involves combining all \glspl{object file} into a single binary, resolving symbols, as well as several more involved processes.
}
\longnewglossaryentry{live}{name={Linking},text={live}}{
  A value is said to be live at a \gls{point} in a program if there exists at least one path from this point to a use of the value.
}
\longnewglossaryentry{llvm}{name={LLVM},text={LLVM}}{
  The name of an umbrella project of compiler technologies including an \gls{ir}, debugger, a variety of machine backends, \etc
  May also be used to refer specifically to the combination of optimiser and backend as part of the \gls{lowering} process. See \rsec{llvm}.
}
\longnewglossaryentry{lowering}{name={Lowering},text={lowering}}{
  The process of transforming one representation of a program to another representation that encodes the same semantics but is closer to a final product.
  Typically begins with source code in a high level language and ends with binary machine code; transitional steps often include a language agnostic \gls{ir} and an \gls{isa} specific assembly listing.
}
\newacronym[
  description={
    Optimisations performed after compilation when \gls{linking} compilation units together.
    Often achieved through ``fat'' binaries, which contain both compiled code and \gls{ir} code, allowing linkers a complete view of all code, including libraries, which is unavailable while compiling individual compilation units.
  }
]{lto}{LTO}{Link-Time Optimisation}

% M
\longnewglossaryentry{mangle}{name=Mangle,text={mangle}}{
  The act of encoding extra information into a string such that the result is unique and the information recoverable.
}
\longnewglossaryentry{matrix engine}{name=Matrix ,text={matrix engine}}{
  A general term describing a set of new facilities on the most recent generation of \glspl{cpu} which focuses on matrix operations.
}
\newacronym[
  description={
    An extension to the \gls{power} \gls{isa}'s \gls{simd} capabilities introduced in \gls{power10}.
    See \rsec{mmaintro}.
  }
]{mma}{MMA}{Matrix Math Assist}

% O
\longnewglossaryentry{object file}{name=Object File,text={object file}}{
  The result of compiling a \gls{translation unit} into a binary.
  Compilers may skip this and directly generate an executable if a program consists of a single translation unit.
}

% P
\longnewglossaryentry{point}{name={Point},text={point}}{
  A point is a concept in program analysis defining places where a program can be paused and the state can be examined.
  A point exists between each logical statement (e.g. between instructions in assembly, between statements in a higher level language) as well as before and after control flow split and join points.
}
\longnewglossaryentry{power}{name={POWER},text={POWER}}{
  \gls{ibm}'s high performance computing architecture built for \gls{hpc} and other applications.
}
\longnewglossaryentry{power10}{name={POWER10},text={POWER10}}{
  The tenth generation of \gls{ibm}'s \gls{power} architecture announced in August 2020.
}
\newacronym[
  description={
    An old name for the \gls{power} \gls{isa} which is still in use in some contexts as a shorthand.
  }
]{ppc}{PPC}{Power PC}

% R
\newacronym[
  description={
    A computer focused on a small, simple, and optimised instruction set meant to enable the processor's pipeline to have a low \gls{cpi}.
  }
]{risc}{RISC}{Reduced Instruction Set Computer}

% S
\longnewglossaryentry{rematerialisation}{name={Rematerialisation},text={rematerialisation}}{
  The process of computing a value multiple times rather than leaving it in a register or \glslink{spill}{spilling} it.
}
\newacronym[
  description={
    Used to describe an architecture with data-level parallelism capabilities, \ie able to perform the same operation on multiple units of data simultaneously.
  }
]{simd}{SIMD}{Single Instruction, Multiple Data}
\newacronym[
  description={
    An \gls{integrated circuit} which is composed of all or most components of a computer (\eg \gls{cpu}, memory, signal processors).
    This is in opposition to a motherboard-based architecture where components may be swapped and are connected via the interfacing motherboard.
  }
]{soc}{SoC}{System on a chip}
\longnewglossaryentry{spill}{name={Spill},text={spill}}{
  The act of storing an in-use register to memory temporarily in order to free the register for another value.
}
\newacronym[
  description={
    A programming paradigm common to \glspl{ir} where each variable in a program is assigned to a single time in the program text.
    This format makes reasoning about code much easier for many optimisations.
    See \rsec{ir}.
  }
]{ssa}{SSA}{Static Single Assignment}
\newacronym[
  description={
    An extension to ARM's Neon vector extension which enables programs to run on hardware with differing vector length without recompilation.
  }
]{sve}{SVE}{Scalable Vector Extension}

% T
\newacronym[
  description={
    The hardware cache responsible for aiding fast virtual to physical address translation.
    Typically implemented as a key-value map where the key is a page number and the value is the frame in which that page is resident.
  }
]{tlb}{TLB}{Translation Lookaside Buffer}
\newacronym[
  description={
    An external \gls{asic} for accelerating AI designed by Google~\autocite{abadi2016tensorflow}.
  }
]{tpu}{TPU}{Tensor Processing Unit}
\longnewglossaryentry{translation unit}{name={Translation Unit},text={translation unit}}{
A single input file to a compiler after the preprocessing step.
All \code{include} directives have been resolved and replaced and preprocessor commands processed.
}

% V
\longnewglossaryentry{vectorisation}{name={Vectorisation},text={vectorisation}}{
  The process of transforming a loop in order to perform multiple iterations simultaneously in a \gls{simd} architecture.
}
\newacronym[
  description={
    An extension to the \gls{power} \gls{isa} implementing the \gls{altivec} standard.
  }
]{vmx}{VMX}{Vector Multimedia Extension}
\newacronym[
  description={
    An extension to the \gls{power} \gls{vmx} enabling even greater \gls{simd} capabilities.
    \todo{VSX and VMX aren't sorted correctly for some reason\ldots}
  }
]{vsx}{VSX}{Vector Scalar Extension}
