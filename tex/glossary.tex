% A
\longnewglossaryentry{altivec}{name={AltiVec},text={AltiVec}}{
  A \gls{simd} instruction set designed for use with single-precision floating point and integer values.
}

% C
\newacronym[
  description={
    The number of CPU clock cycles required to complete an instruction.
    Often used as a performance metric where it refers to the ratio of total instructions executed to total clock cycles.
  }
]{cpi}{CPI}{Cycles Per Instruction}

% H
\newacronym[
  description={The area of computing concerned with performing large scale computations in the most efficient way possible through the minimization of required time and resources.}
]{hpc}{HPC}{High Performance Computing}

% I
\newacronym[
  description={An American multinational technology company with a long history in the computing domain.}
]{ibm}{IBM}{International Business Machines}
\newacronym[
  description={A language-agnostic representation of a program encoding all of the semantics of the original program and potentially annotated with debugging and optimisation information.}
]{ir}{IR}{Intermediate Representation}
\newacronym[
  description={The abstract model of a computer, including registers, supported datatypes, instructions, \etc}
]{isa}{ISA}{Instruction Set Architecture}

% L
\longnewglossaryentry{llvm}{name={LLVM},text={LLVM}}{
  The name of an umbrella project of compiler technologies including an \gls{ir}, debugger, a variety of machine backends, \etc
  May also be used to refer specifically to the combination of optimiser and backend as part of the \gls{lowering} process.
}
\longnewglossaryentry{lowering}{name={Lowering},text={lowering}}{
  The process of transforming one representation of a program to another representation that encodes the same semantics but is closer to a final product.
  Typically begins with source code in a high level language and ends with binary machine code; transitional steps often include a language agnostic \gls{ir} and an \gls{isa} specific assembly listing.
}

% M
\newacronym[
  description={
    An extension to the \gls{power} \gls{isa}'s \gls{vsx} facility introduced in \gls{power10}.
    See \rsec{mmaintro}.
  }
]{mma}{MMA}{Matrix Math Assist}

% P
\longnewglossaryentry{power}{name={POWER},text={POWER}}{
  \gls{ibm}'s high performance computing architecture built for \gls{hpc} and other applications.
}
\longnewglossaryentry{power10}{name={POWER10},text={POWER10}}{
  The tenth generation of \gls{ibm}'s \gls{power} architecture announced in August 2020.
}

% R
\newacronym[
  description={A computer focused on a small, simple, and optimized instruction set meant to enable the processor's pipeline to have a low \gls{cpi}.}
]{risc}{RISC}{Reduced Instruction Set Computer}

% S
\newacronym[
  description={Used to describe an architecture with data-level parallelism capabilities, \ie able to perform the same operation on multiple units of data simultaneously.}
]{simd}{SIMD}{Single Instruction, Multiple Data}
\longnewglossaryentry{spill}{name={Spill},text={spill}}{
  The act of storing an in-use register to memory temporarily in order to free the register for another value.
}

% V
\longnewglossaryentry{vectorisation}{name={Vectorisation},text={vectorisation}}{
  The process of transforming a loop in order to perform multiple iterations simultaneously in a \gls{simd} architecture.
}
\newacronym[
  description={An extension to the \gls{power} \gls{isa} implementing the \gls{altivec} standard.}
]{vmx}{VMX}{Vector Multimedia Extension}
\newacronym[
  description={An extension to the \gls{power} \gls{vmx} enabling even greater \gls{simd} capabilities.}
]{vsx}{VSX}{Vector Scalar Extension}
