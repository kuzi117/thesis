% A workaround to allow relative paths in included subfiles
% that are to be compiled separately
% See https://tex.stackexchange.com/questions/153312/subfiles-inside-a-subfile-using-relative-paths
\providecommand{\main}{..}
\documentclass[\main/thesis.tex]{subfiles}

\begin{document}

% environment for preface
% If you want the preface to not be double-spaced, uncomment the corresponding
% setting above.
\begin{preface}

\rcha{related} and \rsec{alternateLowering} contain content that has been extracted from a multi-author manuscript that is currently being revised~\autocite{kuzma2021fast}.
The portions of \rcha{related} that have been used in this work were drafted by a co-author and edited by all authors.
Due to the large amount of overlap between related sources, only a small amount of editing was performed to make the statements more relevant to this manuscript.
Several new sources not in the multi-author document have also been added in \rcha{related}.

The content in \rsec{alternateLowering} that was duplicated from the original manuscript was written and drafted by myself before being edited by myself and my co-authors.
The version contained in this thesis contains additional details in the description of the code-generation design and additional experimental results and analysis that are not included in the multi-author manuscript.
The plan at the time of writing is for the manuscript to be further revised and to be submitted to a major international journal in the area of code generation.

This multi-author effort focuses on creating a more efficient compiler-only code-generation path for \gls{gemm} in the \gls{llvm} framework.
The work in that manuscript addresses the much broader issue of efficient and large matrix multiplication on multiple platforms with varying memory and hardware constraints.
My contribution to that work, the same contribution described in much greater detail in this thesis, is only one facet of the investigation developed by myself and my co-authors.
Overall, my contribution to that work is the implementation of an efficient and performant matrix-multiplication kernel that makes use of \gls{mma}, allowing for the examination and comparison of the effects of \glspl{matrix engine} on large-scale matrix-multiplication kernels.

% A preface is required if you need to describe how parts of your thesis were
% published or co-authored, and what your contributions to these sections were.
% Also mention if you intend to publish parts of your thesis,
% or have submitted them for publication.
% It is also required if ethics approval was needed for any part of the thesis.
%
% Otherwise it is optional.
%
% See the FGSR requirements for examples of how this can look.
\end{preface}

\end{document}
